\documentclass[a4paper,11pt] {article.cls}%{scrreprt}
\RequirePackage{FFHS}

\dokumentTyp{Semesterarbeit}
\studiengang{Java Enterprise Application}

\author{Peter Müller}
\title{Verkauf innovativer Ideen} %
\subtitle{Online Shop} % optional
\titelbild[width=10cm]{images/title-image}

\wohnort{Bernstrasse 47, 3422 Kirchberg}

% \referent oder \referentin
\referent{Daniel Senften\\Master of Science (MSc)\\Java Enterprise Edition}
\eingereichtBei{Daniel Senften\\Master of Science (MSc)\\Dozent}

\addbibresource{literature.bib} % Hinzufügen der Referenzen

\begin{document}

    \section{Aufgabenstellung}\label{sec:aufgabenstellung}

    Ausgehend von der Beispielanwendung aus dem Lehrbuch entwickelt jeder Studierende einen
    eigenen Onlineshop. Dabei bearbeitet er ein individuelles Fokus-Thema, welches er in Absprache
    mit dem Dozenten waehlt.

    Der Onlineshop, basierend auf der JavaEE\cite{jee-spec} Spezifikation muss insgesamt funktionsaehig und dokumentiert sein.

    Das Hauptgewicht der Semesterarbeit liegt auf der theoretischen Behandlung, der Analyse, dem
    Design, der Implementation und dem Test des ausgeaehlten Fokus-Themas.


    % ==============================
    % Anhang
    \clearpage\appendix
    \renewcommand{\thesection}{A}
    %! suppress = MissingImport
\section{Literatur und Quellenverzeichnis}
\label{sec:references}

\printbibliography[heading=subbibliography,title={~}] % Do not print any additional title


\end{document}
